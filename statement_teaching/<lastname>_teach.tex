\documentclass{jmstatement}

\begin{document}

\pagestyle{fancy}
\fancyhf{}
\fancyhead[L]{\bfseries\scshape \myname}
\fancyhead[R]{\bfseries\scshape Teaching Statement}
\fancyfoot[C]{\thepage}

%------------------------------------------------------------------------------%
\newcommand{\fullname}{Levi G. Crews}
\newcommand{\myname}{Levi Crews}
\newcommand{\myDOB}{8 February 1996}
\newcommand{\mycitizenship}{USA}
\newcommand{\myphone}{+1 (585) 472-6012}
\newcommand{\myemail}{\url{lgcrews@uchicago.edu}}
\newcommand{\mywebsite}{\url{levicrews.com}}
%------------------------------------------------------------------------------%
\newcommand{\capfieldlist}{Macroeconomics, Spatial Economics, International Trade}
\newcommand{\fields}{macroeconomics, spatial economics, and international trade}
\newcommand{\onesenID}{how the spatial distribution of economic activity at different spatial scales---from the urban to the regional to the international---affects macroeconomic aggregates and welfare in the long run}
\newcommand{\threesenID}{My primary research fields are \fields, with a focus on growth and development as well as applications in environmental economics. My research agenda addresses \onesenID. To that end, I develop new dynamic spatial models that I quantify with rich microdata in order to answer policy-relevant questions.}
%------------------------------------------------------------------------------%

% See the included "notes_deib.org" file.
% Org is just another markup language (like Markdown), so you can open it with any text editor.

% Make sure to replace the path to your .bib file below!

Learning how to think like an economist is hard.
We discourage and exclude students when we say anything to the contrary:
nothing is ``obvious'' or ``trivial'' to a novice.
We do them harm if we perpetuate the illusion that some people simply have a
knack for economics whereas others do not, or that experts are magicians who
can understand or do things that others never could.
Our job as teachers and advisors is not to make learning economics easy;
it is to make it exciting and worthwhile even though it is hard.

Luckily, at many colleges and universities, economics is already one of the
(if not \textit{the}) most popular majors, which suggests that many students
enter college with some desire to think like an economist.
But for most, they do so with only a novice's grasp of what economics is.
As \citet{Becker1971} put it, economics is ``the study of the allocation of
scarce means to satisfy competing ends.''
After hearing this definition, students are more likely to see opportunities
to think like an economist everywhere.
As undergraduate instructors, we can reinforce this by discussing stories in
the \textit{New York Times} or the \textit{Wall Street Journal}.
We can even work through, as \citet{Michael2016} does, how many of the biggest
life decisions our students will ever make---from what career to pursue to how
to parent---can be guided by economic principles.

Our job becomes more difficult at the level of individual lessons:
how do we teach the classic two-sector Ricardian model and the Solow growth model
to undergraduates, or the \citet{EatonKortum2002} model and the latest endogenous
growth models to more advanced students?
I think the first step must be to establish the puzzle or fact that a particular
model was developed to address.
The volume of global trade exceeded \$20 trillion last year---what factors explain
the patterns of who trades with whom?
Variation in levels and growth rates of GDP per capita persist across countries---what
mechanisms drive this variation, and can we target them with policies in order
to close the gap?
If we get students interested in a puzzle or an unexplained fact,
then they will care about our (admittedly partial) solutions and explanations.
More generally, if we get them asking sharp questions about the world around them,
they will be better students and citizens for it.

The second step is to figure out what the students already know.
Each student enters a classroom with a unique history of training in economics,
mathematics, statistics, and other related subjects.
Moreover, two students that took the same courses may have different levels of
mastery of the material or have had different emotional experiences working
through the material.
Accordingly, students enter the classroom with a variety of mental models, each of
which needs to be rewired in different ways to match the model we are trying to teach.
As \citet{Polya2014} advised, ``The best [way for the teacher to help their student]
is to help the student naturally. The teacher should put himself in the student's
place, he should see the student's case, he should try to understand what is
going on in the student's mind, and ask a question or indicate a step that could
have occurred to the student himself.''
Rewiring students' mental models requires diagnostic questions (that is,
\textit{formative}, rather than just \textit{summative}, assessments) to see
where students' understanding breaks down \citep{Wilson2019}.
This is where it is important to remember that our job is not to pack our students'
minds with facts and toy models; rather, it is to teach them to think differently
about the world, which is saturated with economics.
Only once we have convinced them of the problems (step 1) and surveyed their
existing mental models (step 2) can we help build the mental scaffolding with
which they can make sense of the coursework and the world around them.

The third step is to design each lesson, and the course as a whole,
with the end in mind. I decide up front: what do I want my students to be able
to \textit{do} at the end of this?
I then design the curriculum working backward from this point.
By focusing on what I want students to be able to \textit{do} rather than on
what I want them to \textit{know},
I can better ensure that students leave my course ready to apply the material.
In my experience, an emphasis on doing also breeds confidence and curiosity in
students: both ``I can do this!'' and ``What \textit{else} can I do?''
become more common refrains.

For undergraduates, my aim is two-fold:
that for any question they have within the purview of the course's material,
they would be able to
(i) gather and interpret relevant data and to
(ii) select and apply an appropriate model, understanding how that model works
and in what situations it breaks down.
In the problem set questions I wrote for Felix Tintelnot's undergraduate trade elective,
for example,
I had students reexamine classic results in trade theory
(e.g., the Stolper-Samuelson theorem)
under different model assumptions so that they could learn how robust or fragile
the results were.

For more advanced students pursuing research, I want them to be able to contribute
to the research frontier.
To be sure, that requires covering foundational and recent papers in the field,
but it also means giving students opportunities to wrangle data, write code,
derive theoretical results, and altogether get their hands dirty.
In the two assignments I wrote for Esteban Rossi-Hansberg's Ph.D. elective in
spatial economics, I did this by walking students through the process of
formulating, quantifying, solving, and applying spatial models.
Many students from the course have continued to work on their models as part
of their research agendas, which I consider as a success for the assignments.
% The same goes for my research assistants: their job requires getting their hands
% dirty already, but I will make sure to mix in tasks that are aimed at their
% development as researchers, like \ldots

% Problem, Explanation, Theory, Example
% - Describe the problem the lesson will solve
% - Work through a solution to that problem
% - Explain the general theory that underpins that solution
% - Work through a second example so that learners will understand which parts generalize

In future courses, I will continue to place an emphasis on doing.
Instead of always giving a traditional chalk talk or slide deck, I will incorporate
demonstrations into my lectures.
If I am teaching a lesson on the gravity model of trade, for instance,
I could begin the lecture by stating the goal of estimating the trade elasticity.
I would then turn it over to the class to ask questions:
Where should I look for data?
What variables do I need?
What assumptions must I impose?
What regression should I run?
In a lesson on the \citet{EatonKortum2002} model, I would have students work
in groups to explore alternative assumptions or small extensions.
For advanced students, I would walk them through the replication of one of my
papers. To teach them how to write well, I would give them the body
of one of my papers and have them rewrite the introduction from scratch.
Between classes, I would ask students to write summaries of what we discussed
in the previous lecture, which I will elaborate, emphasize, or correct as necessary
to ensure students are internalizing the main themes of the course.

As a faculty member, I feel prepared to teach a wide range of undergraduate courses,
including economic principles, intermediate macroeconomics, and field courses
in international trade, spatial economics, and economic growth.
At the introductory level, I would be particularly interested in teaching a course
or seminar based on \citet{Michael2016} that taught students how to look at data
and make decisions like an economist.
At the graduate level, I feel prepared to teach within the first-year macroeconomics
sequence, or to teach field courses in international trade, spatial economics,
and economic growth.
I would also be excited to help develop and teach a course akin to
\href{https://missing.csail.mit.edu/}{MIT's Missing Semester},
covering
when to use common modeling assumptions and how to derive them,
how to organize your notes and research ideas,
how to structure papers and slide decks,
how to write clean code that facilitates replication,
and the like.

\begingroup
\renewcommand{\section}[2]{}%
\bibliographystyle{aer}
\bibliography{../../../Dropbox/crewsbib/crewsbib}
\endgroup

\end{document}