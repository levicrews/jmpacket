\documentclass{jmstatement}

\begin{document}

\pagestyle{fancy}
\fancyhf{}
\fancyhead[L]{\bfseries\scshape \myname}
\fancyhead[R]{\bfseries\scshape Research Statement}
\fancyfoot[C]{\thepage}

%------------------------------------------------------------------------------%
\newcommand{\fullname}{Levi G. Crews}
\newcommand{\myname}{Levi Crews}
\newcommand{\myDOB}{8 February 1996}
\newcommand{\mycitizenship}{USA}
\newcommand{\myphone}{+1 (585) 472-6012}
\newcommand{\myemail}{\url{lgcrews@uchicago.edu}}
\newcommand{\mywebsite}{\url{levicrews.com}}
%------------------------------------------------------------------------------%
\newcommand{\capfieldlist}{Macroeconomics, Spatial Economics, International Trade}
\newcommand{\fields}{macroeconomics, spatial economics, and international trade}
\newcommand{\onesenID}{how the spatial distribution of economic activity at different spatial scales---from the urban to the regional to the international---affects macroeconomic aggregates and welfare in the long run}
\newcommand{\threesenID}{My primary research fields are \fields, with a focus on growth and development as well as applications in environmental economics. My research agenda addresses \onesenID. To that end, I develop new dynamic spatial models that I quantify with rich microdata in order to answer policy-relevant questions.}
%------------------------------------------------------------------------------%

% A good research statement can be quite formulaic:
% - three-sentence ID
% - one paragraph that just rephrases your JMP abstract
% - one paragraph with future work from your JMP conclusion
% - one paragraph each for your other projects

% Make sure to replace the path to your .bib file below!

\threesenID

% This is just the abstract of my JMP!
For example, although cities have long been thought to drive economic growth, analyses
of spatial policies have largely ignored the effects of such policies on growth.
In my job market paper, \textbf{``A Dynamic Spatial Knowledge Economy''},
I develop a spatial endogenous growth model in which
heterogeneous agents make forward-looking migration decisions and
human capital investments over the life cycle.
Local externalities in the human capital investment technology
drive both agglomeration and growth.
I show that, along a balanced growth path, the growth rate depends on the spatial
distribution of human capital, making it sensitive to spatial policies.
I calibrate the model to data on U.S. metropolitan areas
and show that it can rationalize the faster wage growth of workers in big cities,
as well as other key patterns in life-cycle wage profiles, migration decisions,
and city characteristics.
Because workers accumulate human capital at different rates depending
on where they live, the model provides an environment in which spatial policy
can not just attract skilled workers, but produce them, too.
I find that policies that further concentrate skilled workers in large cities
are growth-enhancing.

% Then I pull the "future work" from the conclusion of my JMP.
I am currently working
on additional applications and extensions of this framework.
One application is an ex-ante policy evaluation of
the \$11 billion regional technology hubs program funded through
the American Rescue Plan and the CHIPS and Science Act, which aims to create
``regional growth clusters" outside of the existing coastal hubs.
The results of my job market paper, however, suggest that dispersing skilled labor
acoss more, smaller clusters may actually be counterproductive.
In one extension, I introduce a consumption-savings choice with liquidity constraints
so that some workers ``borrow'' by moving to a cheaper location with slower
wage growth \citep{BilalRossi-Hansberg2021}.
The model provides a channel through which credit availability would have
lasting effects on lifetime human capital accumulation and, thus, aggregate growth.
In another extension,
I am working on how to solve for transitions, which I will then use to study
optimal placed-based development policies in contexts outside of the U.S.
\citep{DurantonVenables2018}.

% Now I pivot to my other ongoing projects, starting with the one that's furthest along.

% This first sentence is all I need to bridge from my JMP to my other projects.
In the rest of my ongoing research, I apply dynamic spatial models to questions
in environmental, agricultural, and resource economics.
% Abstract for my other working paper.
Over 90\% of global water use occurs in agricultural production,
which is subject to two pervasive distortions: (i) incomplete property
rights for farmers accessing water and (ii) subsidies, taxes, and
tariffs affecting agricultural output.
In \textbf{``Agriculture, Trade, and the Spatial Efficiency of Global Water Use''},
I work with Tamma Carleton and Ishan Nath to combine a rich
collection of global geospatial data with a dynamic spatial equilibrium
model in order to quantify the impact of agricultural and trade policies on
regional water scarcity and welfare. In the data, we show that
water-intensive crops concentrate in water-abundant locations,
implying a strong role for comparative advantage in governing global
water use, though a small number of regions with very water-intensive
production are losing water rapidly over time. The model captures
production, consumption, and trade in agriculture across many countries
and crops, as well as the dynamic evolution of local water stocks as
farmers extract water from a common pool. We calibrate the model to
match observed patterns of agricultural production and hydrological
trends, and will use it to compare the existing allocation to the global
planner's undistorted steady state and a hypothetical scenario with no
international trade in agriculture.

% Abstract for work-in-progress #1
In our paper \textbf{``Does Eating Local Reduce Emissions''}, Ishan Nath and I
examine the conventional wisdom that promoting
consumption of locally-produced food reduces greenhouse gas emissions.
We start by exploring the partial equilibrium consequences of a single
consumer's sourcing decisions using existing data on emissions from
shipping, along with a new high-resolution global spatial dataset
containing scientific estimates of crop-wise emissions from agricultural
production. Initial exploration suggests that the spatial variation in
production emissions from agriculture is substantial relative to the
emissions from shipping. Next, we will use a global model of production,
consumption, and trade in agriculture to investigate the general
equilibrium consequences of varying the level of globalization. We plan
to use the model to compare global agricultural emissions under existing
policy to a scenario that imposes autarky on all local regions, and to
an alternative scenario with much greater openness to trade.

% Abstract for work-in-progress #2
In another paper \textbf{``Trade Policy and Food Security''}, Ishan Nath and I
investigate how trade policy affects stability in
food supply and food prices. We show that openness to trade exerts two
competing forces on volatility: (i) diversifying supply across many
countries reduces the exposure of local consumers to domestic or
regional shocks and (ii) relying on imports for consumption of a
necessary good creates vulnerability to geopolitical risk or trade
barriers erected in response to instability. We use global panel data on
agricultural production, prices, trade flows, trade policy, and weather
to examine how trade barriers respond endogenously to agricultural
supply shocks and explore the domestic and international transmission of
price fluctuations. We plan to use a model of production, consumption,
and trade in agriculture to study optimal trade policy for promoting
food supply stability in countries facing endogenous trade barriers and
stochastic shocks to productivity.

% You don't even need a conclusion!

\begingroup
\renewcommand{\section}[2]{}%
\bibliographystyle{lgc-jpe}
\bibliography{../../../Dropbox/crewsbib/crewsbib}
\endgroup

\end{document}