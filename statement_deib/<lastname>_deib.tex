\documentclass{jmstatement}

\begin{document}

\pagestyle{fancy}
\fancyhf{}
\fancyhead[L]{\bfseries\scshape \myname} %
\fancyhead[R]{\bfseries\scshape DEIB Statement} %
% \fancyfoot[C]{\thepage}

%------------------------------------------------------------------------------%
\newcommand{\fullname}{Levi G. Crews}
\newcommand{\myname}{Levi Crews}
\newcommand{\myDOB}{8 February 1996}
\newcommand{\mycitizenship}{USA}
\newcommand{\myphone}{+1 (585) 472-6012}
\newcommand{\myemail}{\url{lgcrews@uchicago.edu}}
\newcommand{\mywebsite}{\url{levicrews.com}}
%------------------------------------------------------------------------------%
\newcommand{\capfieldlist}{Macroeconomics, Spatial Economics, International Trade}
\newcommand{\fields}{macroeconomics, spatial economics, and international trade}
\newcommand{\onesenID}{how the spatial distribution of economic activity at different spatial scales---from the urban to the regional to the international---affects macroeconomic aggregates and welfare in the long run}
\newcommand{\threesenID}{My primary research fields are \fields, with a focus on growth and development as well as applications in environmental economics. My research agenda addresses \onesenID. To that end, I develop new dynamic spatial models that I quantify with rich microdata in order to answer policy-relevant questions.}
%------------------------------------------------------------------------------%

% See the included "notes_deib.org" file.
% Org is just another markup language (like Markdown), so you can open it with any text editor.

I am committed to promoting diversity, equity, inclusion, and belonging
through my research, teaching, and service.
My commitment starts by recognizing my privilege.
I am a white heterosexual cisgender American male.
As such, I have felt comfortable in most seminars and classrooms, which
are populated by many others who look like me and speak my native language.
When I set goals for my career, I can look up to and seek the mentorship
of any number of economists that share my background, even just within my own
department.
Such is not the case for many students from underrepresented groups.

To help address this, I have served in multiple roles in my time as a graduate
student. First, in my third and fourth years I mentored two first-year students
as they adjusted to life as graduate students. I helped them navigate coursework,
relationships with peers and faculty, and the general stresses of starting a Ph.D.
in a new city. I made concerted efforts to seek them out for coffee throughout the
year, to listen well, and to encourage them in the face of tough feedback from
faculty. Both my mentees were English-speaking men, but in the future I hope
to expand my mentorship to students from underrepresented groups.

Second, also in my third and fourth years, I organized two student seminars.
In my work as an organizer, I prioritized fostering an environment in which all
students felt comfortable sharing early stage work.
Sadly, based on my discussions with other students,
the argumentative style of economics seminars can be particularly
discouraging for those from underrepresented groups and those struggling
with their mental health.
I encouraged attendees to offer positive feedback during seminars and to offer
all critical feedback with empathy.
I modeled this behavior to the best of my ability and in doing so developed
a reputation as a sharp but kind reviewer.

Third, in my fourth and fifth years I helped found the Graduate Student Representatives
body within the economics department, through which I served as my cohort's representative.
As my cohort's representative, I advocated for the needs of my peers in front
of the faculty and administration during the Covid-19 pandemic.
Our accomplishments included setting up a department-wide Slack channel to keep
students connected while we were shut out of Saieh Hall,
establishing a series of ``What I Wish I Knew'' lectures with alumni and faculty
to help reveal the hidden curriculum,
and rewriting portions of the student guidebook to be more clear and accomodating
to students in different circumstances.
With many classmates sheltering-in-place away from Chicago and dealing with
personal or family health issues, I made sure to stay in touch with them all
so that I could ensure their equitable treatment by the department.

I have not had the opportunity to design my own courses, since most of my
experience has been as a TA for other Ph.D. students. That said, as a faculty
member I will be committed to providing the same opportunities to students that
show the same potential regardless of color, class, creed, or identity. When
evaluating potential, I am committed to doing so wholistically, accounting for
the opportunities and disadvantages that each student has faced. Rather than
merely advising students, I commit to \textit{sponsoring} them, which means I
will actively seek out opportunities for them to progress toward their career
goals. For students from underrepresented groups, this means being aware of
opportunities like the AEA Summer Training Program and the CSWEP Mentoring Program.
It also means promoting their work and inviting them into both formal and informal
collaborations. Finally, I commit to listening to my students and colleagues
from different backgrounds so that I can continue to learn how best to encourage
them as researchers and people.

\end{document}