\documentclass[margin,line]{res}
\usepackage{hyperref}
\hypersetup{colorlinks=true,urlcolor=blue}
\hypersetup{
	pdftex,
    pdfauthor={},               % author
    colorlinks=true,            % false: boxed links; true: colored links
    linkcolor=red,              % color of internal links
    citecolor=black,            % color of links to bibliography
    urlcolor=blue               % color of external links
}

\oddsidemargin=-.5in
\evensidemargin=-.5in
\textwidth=6.0in
\itemsep=0in
\parsep=0in
% if using pdflatex:
\setlength{\pdfpagewidth}{\paperwidth}
\setlength{\pdfpageheight}{\paperheight}

\newenvironment{list0}{
  \begin{list}{$\bullet$}{%
      \setlength{\itemsep}{0in}
      \setlength{\parsep}{0in} \setlength{\parskip}{0in}
      \setlength{\topsep}{0in} \setlength{\partopsep}{0in}
      \setlength{\leftmargin}{0.0in}}}{\end{list}}

\newenvironment{listtab}{
\begin{list}{$\bullet$}{%
    \setlength{\itemsep}{0in}
    \setlength{\parsep}{0in} \setlength{\parskip}{0in}
    \setlength{\topsep}{0in} \setlength{\partopsep}{0in}
    \setlength{\leftmargin}{0.075in}}}{\end{list}}

\newenvironment{list1}{
  \begin{list}{$\bullet$}{%
      \setlength{\itemsep}{0in}
      \setlength{\parsep}{0in} \setlength{\parskip}{0in}
      \setlength{\topsep}{0in} \setlength{\partopsep}{0in}
      \setlength{\leftmargin}{0.17in}}}{\end{list}}

\newenvironment{list2}{
  \begin{list}{$\bullet$}{%
      \setlength{\itemsep}{0in}
      \setlength{\parsep}{0in} \setlength{\parskip}{0in}
      \setlength{\topsep}{0in} \setlength{\partopsep}{0in}
      \setlength{\leftmargin}{0.2in}}}{\end{list}}

%------------------------------------------------------------------------------%
\newcommand{\fullname}{Levi G. Crews}
\newcommand{\myname}{Levi Crews}
\newcommand{\myDOB}{8 February 1996}
\newcommand{\mycitizenship}{USA}
\newcommand{\myphone}{+1 (585) 472-6012}
\newcommand{\myemail}{\url{lgcrews@uchicago.edu}}
\newcommand{\mywebsite}{\url{levicrews.com}}
%------------------------------------------------------------------------------%
\newcommand{\capfieldlist}{Macroeconomics, Spatial Economics, International Trade}
\newcommand{\fields}{macroeconomics, spatial economics, and international trade}
\newcommand{\onesenID}{how the spatial distribution of economic activity at different spatial scales---from the urban to the regional to the international---affects macroeconomic aggregates and welfare in the long run}
\newcommand{\threesenID}{My primary research fields are \fields, with a focus on growth and development as well as applications in environmental economics. My research agenda addresses \onesenID. To that end, I develop new dynamic spatial models that I quantify with rich microdata in order to answer policy-relevant questions.}
%------------------------------------------------------------------------------%

%------------------------------------------------------------------------------%

\begin{document}

\name{\fullname \hfill{} {\small\normalfont Updated: \today} \hspace{6.5em} \vspace*{.1in}}

\begin{resume}

%------------------------------------------------------------------------------%

\section{\sc Contact Information}
\vspace{.05in}
\begin{tabular}{@{}p{2in}p{4in}}
University of Chicago & \myemail \\
Department of Economics & \myphone \\
1126 E 59 Street & \mywebsite \\
Chicago, IL 60637 USA & \\
\end{tabular}

%------------------------------------------------------------------------------%

\section{\sc Personal}
\begin{tabular}{@{}ll}
    Citizenship & \mycitizenship \\
    Date of birth & \myDOB
\end{tabular}

%------------------------------------------------------------------------------%

\section{\sc Research Fields}
\capfieldlist

%------------------------------------------------------------------------------%

\section{\sc Education}
\textbf{University of Chicago}, Chicago, Illinois USA
\vspace*{-2mm}

\begin{tabular}{ll}
    Ph.D. in Economics & 2017 -- 2023 (expected) \\
    M.A. in Economics & 2019 \\
\end{tabular}
\vspace*{.1in}
\begin{list0}
    \item[] {\sc Placement} \\
    \vspace*{-2mm}
    \begin{list1}
        \item[] \textbf{Co-directors}:
        Ufuk Akcigit,
        \url{uakcigit@uchicago.edu},
        +1 (773) 702-0433
        \item[] \phantom{\textbf{Co-directors}:}
        Manasi Deshpande,
        \url{mdeshpande@uchicago.edu},
        +1 (773) 702-8260
        \item[] \textbf{Coordinator}:
        Kathryn Falzareno,
        \url{kfalzareno@uchicago.edu},
        +1 (773) 702-3026
    \end{list1}
    \vspace*{.09in}
    \item[] {\sc References} \\
    \vspace*{-2mm}
    \begin{list1}
        \item[]
        \begin{tabular}{@{}p{3in}p{3in}}
            Esteban Rossi-Hansberg (Chair) & Jonathan I. Dingel \\
            Glen A. Lloyd Distinguished Service Professor & Associate Professor \\
            University of Chicago & Chicago Booth \\
            \url{rossihansberg@uchicago.edu} & \url{jdingel@chicagobooth.edu} \\
            +1 (773) 834-5240 & +1 (773) 834-5458 \\
            & \\
            Fernando Alvarez & Felix Tintelnot \\
            Charles F. Grey Distinguished Service Professor & Associate Professor \\
            University of Chicago & University of Chicago \\
            \url{f-alvarez1@uchicago.edu} & \url{tintelnot@uchicago.edu} \\
            +1 (773) 702-4412 & +1 (773) 702-3478
        \end{tabular}
    \end{list1}
\end{list0}

\textbf{Duke University}, Durham, North Carolina USA\\
\vspace*{-.1in}
\begin{listtab}
    \item[] B.Sc. in Economics (High Distinction), B.A. in Mathematics, \emph{Summa Cum Laude}, 2017
\end{listtab}

%------------------------------------------------------------------------------%

% \section{\sc Publications}

%------------------------------------------------------------------------------%

\section{\sc Working Papers}
\begin{list0}
    \item[] ``A Dynamic Spatial Knowledge Economy'' (\textbf{Job Market Paper})
    \vspace*{2mm}
    \begin{list1}
        \item[] % Abstract
        \noindent
        Cities have long been thought to drive economic growth. Despite this, analyses
        of spatial policies have largely ignored the effects of such policies on growth.
        In this paper, I develop a spatial endogenous growth model in which
        heterogeneous agents make forward-looking migration decisions and
        human capital investments over the life cycle.
        Local externalities in the human capital investment technology
        drive both agglomeration and growth.
        I show that, along a balanced growth path, the growth rate depends on the spatial
        distribution of human capital, making it sensitive to spatial policies.
        I calibrate the model to data on U.S. metropolitan areas
        and show that it can rationalize the faster wage growth of workers in big cities,
        as well as other key patterns in life-cycle wage profiles, migration decisions,
        and city characteristics.
        Because workers accumulate human capital at different rates depending
        on where they live, the model provides an environment in which spatial policy
        can not just attract skilled workers, but produce them, too.
        I find that policies that further concentrate skilled workers in large cities
        are growth-enhancing.
    \end{list1}
    \vspace*{2mm}
    \item[]
    \item[]
    \item[] ``Agriculture, Trade, and the Spatial Efficiency of Global Water Use''
    (with \href{https://www.tammacarleton.com/}{T.~Carleton}
    \& \href{https://www.ishannath.com/}{I.~Nath})
    \vspace{-2mm}
    \begin{list1}
        \item[] Over 90\% of global water use occurs in agricultural production,
        which is subject to two pervasive distortions: (i) incomplete property
        rights for farmers accessing water and (ii) subsidies, taxes, and
        tariffs affecting agricultural output. This paper combines a rich
        collection of global geospatial data with a dynamic spatial equilibrium
        model to quantify the impact of agricultural and trade policies on
        regional water scarcity and welfare. In the data, we show that
        water-intensive crops concentrate in water-abundant locations,
        implying a strong role for comparative advantage in governing global
        water use, though a small number of regions with very water-intensive
        production are losing water rapidly over time. The model captures
        production, consumption, and trade in agriculture across many countries
        and crops, as well as the dynamic evolution of local water stocks as
        farmers extract water from a common pool. We calibrate the model to
        match observed patterns of agricultural production and hydrological
        trends, and will use it to compare the existing allocation to the global
        planner's undistorted steady state and a hypothetical scenario with no
        international trade in agriculture.
    \end{list1}
\end{list0}

%------------------------------------------------------------------------------%

\section{\sc Work In \\ Progress}
\begin{list0}
    \item[] ``Does Eating Local Reduce Emissions?''
    (with \href{https://www.ishannath.com/}{I.~Nath})
    \vspace*{2mm}
    \begin{list1}
        \item[] This paper examines the conventional wisdom that promoting
        consumption of locally-produced food reduces greenhouse gas emissions.
        We start by exploring the partial equilibrium consequences of a single
        consumer's sourcing decisions using existing data on emissions from
        shipping, along with a new high-resolution global spatial dataset
        containing scientific estimates of crop-wise emissions from agricultural
        production. Initial exploration suggests that the spatial variation in
        production emissions from agriculture is substantial relative to the
        emissions from shipping. Next, we will use a global model of production,
        consumption, and trade in agriculture to investigate the general
        equilibrium consequences of varying the level of globalization. We plan
        to use the model to compare global agricultural emissions under existing
        policy to a scenario that imposes autarky on all local regions, and to
        an alternative scenario with much greater openness to trade.
    \end{list1}
    \vspace*{2mm}
    \item[] ``Trade Policy and Food Security''
    (with \href{https://www.ishannath.com/}{I.~Nath})
    \vspace*{2mm}
    \begin{list1}
        \item[] This paper investigates how trade policy affects stability in
        food supply and food prices. We show that openness to trade exerts two
        competing forces on volatility: (i) diversifying supply across many
        countries reduces the exposure of local consumers to domestic or
        regional shocks and (ii) relying on imports for consumption of a
        necessary good creates vulnerability to geopolitical risk or trade
        barriers erected in response to instability. We use global panel data on
        agricultural production, prices, trade flows, trade policy, and weather
        to examine how trade barriers respond endogenously to agricultural
        supply shocks and explore the domestic and international transmission of
        price fluctuations. We plan to use a model of production, consumption,
        and trade in agriculture to study optimal trade policy for promoting
        food supply stability in countries facing endogenous trade barriers and
        stochastic shocks to productivity.
    \end{list1}
    \vspace*{2mm}
    \item[] ``Predicting Trade Elasticities in the US-China Trade War''
    (with \href{http://www.jdingel.com/}{J.~Dingel},
    \href{https://www.sebastianheise.com/}{S.~Heise}, \&
    \href{https://www.felix-tintelnot.com/}{F.~Tintelnot})
\end{list0}

%------------------------------------------------------------------------------%

\section{\sc Presentations}

\begin{list0}
    \item[\phantom{2022}\textbf{2022}] LACEA LAMES, BFI Coase Project,
    UChicago (Capital Theory, Trade \& Spatial working group, Applied Macro Theory lunch)
    \item[\phantom{2021}\textbf{2021}]
    UChicago (Capital Theory, Trade \& Spatial working group, Applied Macro Theory lunch)
\end{list0}
% \vspace*{-.25in}

%------------------------------------------------------------------------------%

\section{\sc Research Assistance}
\textbf{University of Chicago}
\vspace*{-2mm}

\begin{tabular}{ll}
    J.~Dingel and F.~Tintelnot & Jan. 2019 -- June 2020 \\
    H.~Uhlig and D.~Kr{\"u}ger & Feb. 2019 -- June 2020 \\
    B.~Neiman and J.~Vavra & May 2019 -- Nov. 2019
\end{tabular}

%------------------------------------------------------------------------------%

\newpage
\section{\sc Teaching}
\textbf{University of Chicago}
\vspace*{-2mm}

\begin{tabular}{llll}
    TA & Spatial Economics (PhD) & E.~Rossi-Hansberg & Winter 2022 \\
    TA & Theory of Income III (PhD) & F.~Alvarez & Spring 2021 \\
    TA & International Trade (U) & F.~Tintelnot & Winter 2021 \\
    TA & Managing the Firm in the Global Economy (MBA) & J.~Dingel & Winter 2020--21 \\
    TA & Financial Markets in the Macroeconomy (PhD) & V.~Guerrieri & Spring 2020 \\
    TA & International Financial Policy (MBA) & R.~Kekre & Spring 2020
\end{tabular}

\textbf{Duke University}
\vspace*{-2mm}

\begin{tabular}{llll}
    TA & Intermediate Macroeconomics (U) & M.~Connolly & Fall 2016--Spring 2017 \\
    \phantom{TA} & \phantom{Managing the Firm in the Global Economy (MBA)} & \phantom{E.~Rossi-Hansberg} & \phantom{Winter 2020--21}
\end{tabular}

\vspace{-3mm}

%------------------------------------------------------------------------------%

\section{\sc Honors and Awards}

\begin{tabular}{@{}lll}
    Margaret G. Reid Dissertation Fellowship & University of Chicago Economics Department & 2022--23 \\
    Data Acquisition Grant & University of Chicago Economics Department & 2019 \\
    Travel Grant & Princeton Initiative: Macro, Money and Finance & 2019 \\
    Neubauer Fellowship & University of Chicago Social Sciences Division & 2017--22 \\
    Davies Fellowship & Duke University Economics Department & 2016 \\
    Student Marshal & Duke University & 2016 \\
    Phi Beta Kappa & Duke University & 2016 \\
\end{tabular}

%------------------------------------------------------------------------------%

\section{\sc Service}

\begin{tabular}{@{}lll}
    Cohort Representative & 2020--22 \\
    Coordinator: Trade \& Spatial working group & 2020--21 \\
    Peer Mentor & 2019--21 \\
    Coordinator: Applied Macro Theory lunch & 2019--20 \\
\end{tabular}

\vspace*{-1.5mm}

%------------------------------------------------------------------------------%

\section{\sc Referee}
\begin{list0}
    \item[]
    \textit{Journal of Political Economy},
    \textit{Review of Economics and Statistics}
\end{list0}

%------------------------------------------------------------------------------%

\section{\sc Technical \\ Skills}
\begin{list0}
    \item[] Python, Julia, Matlab, Stata, \LaTeX, Unix, Make
\end{list0}

\end{resume}
\end{document}