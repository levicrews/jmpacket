\documentclass{jmstatement}

\setlength{\parindent}{0pt}
\setlength{\parskip}{3ex}

\begin{document}

\pagestyle{fancy}
\fancyhf{}
\fancyhead[L]{\bfseries\scshape \myname}
\fancyhead[R]{\bfseries\scshape Interview Spiel}
\fancyfoot[C]{\thepage}

%------------------------------------------------------------------------------%
\newcommand{\fullname}{Levi G. Crews}
\newcommand{\myname}{Levi Crews}
\newcommand{\myDOB}{8 February 1996}
\newcommand{\mycitizenship}{USA}
\newcommand{\myphone}{+1 (585) 472-6012}
\newcommand{\myemail}{\url{lgcrews@uchicago.edu}}
\newcommand{\mywebsite}{\url{levicrews.com}}
%------------------------------------------------------------------------------%
\newcommand{\capfieldlist}{Macroeconomics, Spatial Economics, International Trade}
\newcommand{\fields}{macroeconomics, spatial economics, and international trade}
\newcommand{\onesenID}{how the spatial distribution of economic activity at different spatial scales---from the urban to the regional to the international---affects macroeconomic aggregates and welfare in the long run}
\newcommand{\threesenID}{My primary research fields are \fields, with a focus on growth and development as well as applications in environmental economics. My research agenda addresses \onesenID. To that end, I develop new dynamic spatial models that I quantify with rich microdata in order to answer policy-relevant questions.}
%------------------------------------------------------------------------------%

% Refer to my advice in panel.pdf

% Refine this so it sounds conversational: It should simultaneously...
% - be so rehearsed that you can pick up from any point with ease if interrupted, and
% - sound as if you're coming up with it naturally on the spot.

% Remember: You want them to get excited about what you do,
%           but a prerequisite is that they understand what you do!

%----------------------------- Introduce yourself -----------------------------%

\textbf{Thank you all for having me here today.} My name is \myname, and in
my work I study \onesenID.

%-------------------------------- 2min spiel ----------------------------------%

\textbf{In my job market paper},
% stating the research question (R), albeit in a somewhat roundabout way
I evaluate a classic hypothesis---namely, that
human capital spillovers attract workers into cities,
and that the human capital accumulation that happens in cities then drives growth.
So, in other words, agglomeration and growth are both outcomes of this process
of knowledge sharing and accumulation.
% narrative propulsion: "but there's a problem" + this is the Position in the RAP
But developing a model in which this hypothesis could even possibly happen---let
alone one that can be tested against the data or used to evaluate policy---has
been a long-standing open problem in spatial economics.
% problem exists "because" (= a "therefore" statement)
The reason is that this problem is inherently high-dimensional:
to have growth, you obviously need dynamics, and you need some kind of learning
or investment that's presumably forward-looking.
But to have cities you obviously need space,
and if those cities attract workers with human capital spillovers, then
space and dynamics are going to be inextricably linked in determining the equilibrium.
% drive it home: everyone should care about this problem (that presumably you're about to solve)
This problem is so difficult and yet important that the
\textit{Handbook of Economic Growth}
called it ``\textit{the} hard problem of regional economics.''
% really lucked out that this problem had been given a name already :sweat_smile:

% State your contribution explicitly, succinctly --> memorably!
My contribution is to tackle the hard problem.
% State how you're going to do it (M)
I do so by adapting tools from heterogeneous agent macro in order to solve
a new quantitative spatial endogenous growth model that I develop.
% Highlight only the most important features of the method
% ask yourself: "What does someone absolutely need to know to understand my results?"
Workers in the model accumulate human capital more quickly in bigger
and more-skilled cities,
% this is part of the answer (A) to the research question (R)
so the model provides a link between the spatial distribution of economic
activity and growth.
That means that any policies that affect that distribution, like place-based policies,
can affect aggregate growth, too.
% Whenever possible, remind people that you're contributing to the literature
So this is the first framework that would allow us to trace out the long-run
growth effects of policies that attract workers to particular locations or
induce them to move away from others.

% more M and A
I quantify the model using data on U.S. metropolitan areas
% highlight the most important empirical contribution, then lump the rest together
and show that it can rationalize the faster wage growth of workers in big cities,
as well as other key patterns in life-cycle wage profiles, migration decisions,
and city characteristics.
% state results of counterfactual or policy exercise as clearly and directly as possible
As one of many possible applications of my framework,
I consider a counterfactual in which I relax land use regulations
in New York and San Francisco to the level of the median city, and I find that
aggregate growth for the entire economy speeds up by 13 basis points.

% Notice that describing M & A mostly follows the sequencing of the paper itself.

%-------------------------------- 5-7min spiel --------------------------------%

% Signposting is crucial in any verbal presentation!
\textbf{So that's an overview of my paper. Now I'd be happy to dive into the details.}
I'll start by talking about the model.
As I mentioned before, I develop and solve a new quantitative spatial endogenous growth model.
% Your job is to anticipate the natural sequence of questions your listener would ask.
% First: "What are the main pieces of the model (environment, agents, decision problem)?"
In particular, there is a system of cities, representing almost every MSA in the U.S.,
populated by heterogeneous workers. Those workers make forward-looking
migration decisions and human capital investments over the life cycle.
% "what's important about how the model's pieces interact?"
A key feature of the environment is a local externality
that affects workers' returns to investment.
Workers accumulate human capital more quickly in bigger and more skilled
cities, which attracts workers to those cities despite higher costs of
living. At the same time, this process of human capital accumulation is
what's driving aggregate growth.
% "so how do you make this tractable?"
What makes this model tractable is that a balanced growth path of this economy can
be written as what's called a ``mean field game."
% "um, OK, what's that?"
That means it can be described by just three things:
\vspace{-2ex}
\begin{enumerate}
    \item a continuous-time Bellman equation to determine workers' decision rules,
    \item a Kolmogorov forward equation to determine how the distribution of workers
    will evolve in response to those rules, and
    \item an expression for the growth rate that depends on this distribution.
\end{enumerate}
\vspace{-2ex}
% "that expression seems key... what is it?"
Specifically, I show that the growth rate can be expressed as a
weighted average of the returns to investment in human capital, with the
investment done in bigger and more skilled cities receiving more weight.
% "can you break that down for me with an example?"
So spending 10\% of your time learning in Chicago contributes more not only
to your own wage growth but also to aggregate growth than the same 10\% in Iowa City.
% "Interesting. What does that imply about the world of your model?"
It follows that the spatial distribution of human capital matters
for aggregate growth in this environment.
Thus, workers' decision rules also matter, as do spatial policies, to which workers respond.
So this is an environment in which spatial policies can matter for growth, and
not even just by moving a fixed stock of skilled workers around the urban system,
but by affecting the very \textit{production} of skilled workers in the economy.

% "Very cool. But why should I think your model represents the real world?"
I then quantify the model using U.S. data. I'll show that the model
can jointly rationalize a host of facts about the urban cross-section---like
the city size distribution and the fact that bigger cities are generally
more productive, more skilled, but also more expensive---as well as facts
about worker panels---like the recent evidence that workers learn more
in big cities as well as observed patterns of life-cycle migration. All the
while, the quantified model generates a 2\% aggregate growth rate as observed
in the postwar U.S. economy.

% "OK, why should I believe your results?"
Let me tell you how I do it. I proceed in three steps.
In the first, I use a five-year sample of migration flows between MSAs from the
ACS to back out migration frictions. A key feature of the data is that I can
separately observe people who move within the same MSA from those who don't move
at all. That allows me to back out what I call the ``opportunity friction''
directly from the observed flows, which governs how often workers can move.
These flows are actually cross-tabbed by age, so I allow that opportunity friction
to vary with age to account for things like marital status, family size,
and homeownership that aren't in the model but affect mobility. I then construct
a Head-Ries index with the flows to back out the bilateral migration costs.

% "Fine, standard enough... what next?"
The second step is to calibrate a subset of the parameters to previous structural
estimates of the Ben-Porath model of human capital accumulation, which is nested
by my model.
% "OK, so you're not estimating the full model? Why not?"
I take this approach primarily because of data limitations: common
US worker panels like the PSID or NLSY have only a few thousand workers, so I
wouldn't have enough power to separately identify location effects even if I
had the computational power to match all the relevant moments in one scheme.

% "Fair enough, but what about your model's key parameters? Hopefully you're not calibrating those..."
In the third step, I use a minimum-distance estimator to pin down the remaining
parameters, chief among them the elasticity that governs the strength of the
human capital spillovers.
I identify this by matching the recent evidence that workers learn more in big cities.
% "I'm not familiar... what's the evidence?"
In particular, what recent work has shown is that there is a sizable dynamic component
to the city size wage premium: When a worker moves from a small city like Cedar
Rapids, Iowa to a big one like Chicago, not only does her wage \textit{level}
jump upon moving, but her wage \textit{growth} speeds up as well. This
additional value of experience \textit{persists} even after moving back to a
smaller city, which suggests that the worker's faster wage growth in fact
reflects faster accumulation of human capital, consistent with my model.
I use the degree to which wage growth speeds up with city size to pin down
the strength of the spillovers in the investment technology, which is the
dynamic agglomeration elasticity.
Separately, I can use the degree to which the wage level jumps upon moving
to pin down a static agglomeration elasticity that affects workers' productivity
but not their human capital accumulation.

% "OK, so you've got a model that fits the data. What can you do with it?"
Once it's quantified, we can use my framework to think about the effects of spatial policy.
% "What kind of policy? Optimal policy?"
The optimal policy is still a work in progress,
so I'll focus on my counterfactual exercise.
I relax land use regulations in two large, skilled cities---New York and San
Francisco---which reduces the congestion elasticity in each one, thereby
making it cheaper for both cities to support larger populations.
% "And what do you find?"
I find that, along the new balanced growth path, both cities are larger and more
skilled, with New York especially so.
% "What's the mechanism?"
This is not primarily the result of skill-biased migration, though, as if the
brain hubs simply syphoned off skilled workers from other cities more than they
did before. Instead, the rate of human capital accumulation in both cities
increases, meaning that the spatial policy helps produce a larger stock of
skilled workers overall.
% "What's the headline number that I should take away?"
Ultimately, I find that the aggregate growth rate for the entire economy
increases by 13 basis points in response to the policy.

%------------------------------- Interview Q's --------------------------------%

\newpage
\textbf{Most common questions about general fit}
\vspace{-2ex}
\begin{itemize}
    \item \textbf{What would you teach?}
    At the undergraduate level, I feel equipped to teach anything, but I would
    most like to teach macro courses in the core sequence and field courses in
    trade, spatial, and growth. At the graduate level, I could teach in the core
    macro sequence and field courses in trade, spatial, and growth.
    \item \textbf{What would the syllabus be for your second-year graduate course?}
    Well, I would want to coordinate with others in the department to minimize
    overlap, but one option I'm excited about would be to construct a course at
    the intersection of trade and spatial economics around the gravity model.
    Two keystone papers would be Allen, Arkolakis, \& Takahashi on universal
    gravity and Redding \& Rossi-Hansberg on quantitative spatial models, but
    then you could cover a bunch of papers at different spatial scales.
\end{itemize}

\end{document}
